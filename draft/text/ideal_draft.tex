\documentclass[12pt]{article}
\usepackage[top=1in, bottom= 1in, left= 1in, right= 1in]{geometry}
\usepackage[USenglish]{babel} % set the language; greek allows \textgreek{\euro}
\usepackage{multirow} % For tables
\usepackage{graphicx, subfigure} % For graphics
\usepackage{fancyhdr} % Produces fancy headers
\usepackage{setspace} % allows for vsape
\usepackage{natbib} % package to organize literature --> google it!
\usepackage{verbatim} % For including R-code
\usepackage{booktabs} % nicer tables
\usepackage{alltt} % verbatim + highlighting
\usepackage{amsmath} %boldsymbols
\usepackage{lscape} %Querformat
\usepackage{dcolumn} % align at decimal mark
\usepackage{floatrow} % description paragraphs below figures and tables
\usepackage{enumerate} % alter enumerate items (i,ii,iii etc)
\usepackage{xcolor}
\usepackage{todonotes}
\usepackage[colorlinks=true,citecolor=red!50!black,urlcolor=black]{hyperref}
\setlength{\headheight}{15pt}
% http://en.wikibooks.org/wiki/LaTeX/Page_Layout for additional info

\author{Peter DeScioli\thanks{\href{mailto:pdescioli@gmail.com}{pdescioli@gmail.com}} \and
		Patrick W. Kraft\thanks{\href{mailto:patrick.kraft@stonybrook.edu}{patrick.kraft@stonybrook.edu}} }
\title{{\small Working Paper}\\
How the Nature of Political Preferences Shapes the Efficiency of Majority Rule Voting \\
{\large Relaxing Assumptions about Voter Utilities}}
\date{Stony Brook University \vspace{1cm}\\ \today}


\begin{document}
\maketitle
\doublespacing



\section{Introduction}

Traditional models of issue voting assume that voters and candidates can be placed on a single policy/ideology dimension and the voters' utilities can be determined by the relative proximity of their ideal points to the respective candidates \citep[c.f.][]{downs1957economic}. In such a framework, simple majority elections between two candidates are generally expected to lead to desirable outcomes that maximize social welfare. The goal of this paper is to examine how the underlying assumption of voter utilities based on common policy dimensions affect the expected welfare outcomes of majority voting.

A large body of research in political science and political sociology showed how a multitude of factors can affect voter preferences independent of pure issue positions and ideological dimensions. Such factors include, but are not limited to, the candidates' traits and personalities, their perceived competence, and the nature of the campaign \citep[see for example][]{hayes2005candidate}. For example, \cite{todorov2005inferences} showed that competence assessments solely based on candidate pictures successfully predicted the results in U.S. congressional elections \citep[see also][]{mattes2010predicting}. Furthermore, the effect of candidate appearance on electoral success is not limited to the related inference about competence, but can be based on simple assessments of the beauty of candidates \citep{berggren2010looks}. These examples of non-issue based determinants of voter preferences indicate the underlying utilities for candidates or parties might not be reducible to a simple issue-based logic.

In the paper presented here, it will be argued, that focusing solely on policy-based utilities induces strong assumptions about the relationships between the utilities for competing candidates. Our goal is to show how relaxing such assumptions can alter our conclusions about the efficiency of voting rules. As a first step, we will focus on a simple voting scenario of two competing candidates and varying sizes of the electorate. We present simulation studies in order to examine the efficiency of majority elections under different scenarios. Based on the simulation results, we propose an experimental design in order to provide further insights as to how the assumptions underlying the ideal-point framework influence the expected social welfare outcomes of voting rules.



\section{Political Preferences and the Ideal Point Framework}

Spatial theories of elections and voter preferences have been very prominent in political science. The  since they have been introduced by \citet{downs1957economic}

Formal models of voting behavior and political representation usually assume that the voters' utilities are

\section{Majority Voting and Social Welfare}

\citet{hastie2005robust}



\section{Simulation Results}




\section{Possible Experimental Approaches}

\citet{oprea2007compensation}

comparing auction mechanism to voting

uncertainty about issue positions




\clearpage

\bibliographystyle{apsr}
\bibliography{/data/Dropbox/1-src/lit/Literature}

\end{document}